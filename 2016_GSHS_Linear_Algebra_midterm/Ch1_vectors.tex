\chapter{Vectors}

% \iffalse 부터 \fi 까지는 주석처리됨
\iffalse
\begin{theorem}[Algebraic Properties of Vectors in \Rn] \label{properties_vectors}
	Let $\vec{u}$, $\vec{v}$, $\vec{w}$ be vectors in \Rn and let $c$ and $d$ be scalars. Then
	\begin{enumerate}
		\item $\vec{u}+\vec{v}=\vec{v}+\vec{u}$ : Commutativity
		\item $(\vec{u}+\vec{v})+\vec{w}=\vec{u}+(\vec{v}+\vec{w})$ : Associativity
		\item $\vec{u}+\vec{0}=\vec{u}$
		\item $\vec{u}+(-\vec{u})=\vec{0}$
		\item $c(\vec{u}+\vec{v})=c\vec{u}+c\vec{v}$ : Distributivity
		\item $(c+d)\vec{u}=c\vec{u}+d\vec{u}$ : Distributivity
		\item $c(d\vec{u})=(cd)\vec{u}$
		\item $1\vec{u}=\vec{u}$
	\end{enumerate}
\end{theorem}

A theorem shown above is thm.\ref{properties_vectors}.

\begin{theorem}
	Let $\vec{u}$, $\vec{v}$, and $\vec{w}$ be vectors in $\mathbb{R}^{n}$ and let $c$ be a scalar. Then
	\begin{enumerate}
		\item $\vec{u}\cdot\vec{v}=\vec{v}\cdot\vec{u}$
		\item $\vec{u}\cdot(\vec{v}+\vec{w}) = \vec{u}\cdot\vec{v}+\vec{u}\cdot\vec{w}$
		\item $(c\vec{u})\cdot\vec{v}=c(\vec{u}\cdot\vec{v})$
		\item $\vec{u}\cdot\vec{u}\geq 0$ and $\vec{u}\cdot\vec{u}=0$ iff $\vec{u}=\vec{0}$
	\end{enumerate}
\end{theorem}
\begin{proof}
	\begin{equation} \label{pi}
	\pi = 3.141592...
	\end{equation}
	By \eqref{pi}, proved! \qedhere
\end{proof}
\fi

\section{Terminology relating to vectors}

A vector can be represented either in geometric way or algebraic way. In geometric definition of vectors, a vector is a \textbf{directed line segment}.
A vector from point $A$ (\textbf{initial point}, or \textbf{tail}) to point $B$ (\textbf{terminal point}, or \textbf{head}) is denoted as $\overrightarrow{AB}$.
Vectors with their tails in the origin is called \textbf{position vectors}, and they are at \textbf{standard position}.
\\
In algebraic view of vectors, a vector is an \textbf{ordered pair} of \textbf{components}. We denote the set of all vectors containing $n$ components in $\mathbb{R}$ as \Rn. Similarly, set of all vectors containing $n$ integer components is $\mathbb{Z}^{n}$.
\\
A vector is written in form of \textbf{column vectors} and \textbf{row vectors}. We use square brackets for denoting vectors, such as 
\begin{displaymath}
\textbf{v} = \begin{bmatrix}4 \\ 1 \\ 6\end{bmatrix},
\begin{bmatrix} 4, & 1, & 6 \end{bmatrix}
\end{displaymath}
\\
A \textbf{zero vector} is a vector which components are all zero. A zero vector is denoted as $\textbf{0}$.
\\
Two vectors are equal if and only if all the components of two vectors are equal. (Of course, the number of components should be same.)
\\
\textbf{Standard unit Vectors} have components which one of them is 1 and rest of them are all 0. Unit vector which has 1 in $i$th component is denoted as $\textbf{e}_i$, and
\begin{displaymath}
\textbf{e}_i = \begin{bmatrix} 0, & 0, & \cdots & 1, & \cdots & 0 \end{bmatrix}
\end{displaymath}
* \textit{Note} We should always denote vector either with arrows ($\vec{v}$), or with boldface letters ($\textbf{v}$). Scalar denotations ($v$) are not allowed.
\\
A set of vectors with $n$ components taken from finite set of integers $\mathbb{Z}_m = \{0, 1, 2,\cdots,m-1\}$ is denoted as $\mathbb{Z}^{n}_{m}$.
$\mathbb{Z}^{n}_{m}$ is closed with respect to operations of vector addition and scalar multiplication (which is defined later). We can perform those operations in same way, but with modulo operations.
Vectors in $\mathbb{Z}^{n}_2$ (all components are 0 or 1) are called \textbf{binary vectors}.

\section{Basic Operations of Vectors}

\begin{theorem}[Algebraic Properties of Vectors : Basic Vector Operations]
Let \textbf{u},\textbf{v}, and \textbf{w} be vectors in \Rn and let $c$ and $d$ be scalars.
    \begin{enumerate}
        \item $\textbf{u} + \textbf{v} = \textbf{v} + \textbf{u}$ (Commutative Property of Vector Addition)
        \item $(\textbf{u} + \textbf{v}) + \textbf{u} = \textbf{u} + (\textbf{v} + \textbf{w})$ (Associative Property of Vector Addition)
        \item $\textbf{u} + \textbf{0} = \textbf{u}$
        \item $\textbf{u} + (-\textbf{u}) = \textbf{0}$
        \item $c(\textbf{u} + \textbf{v}) = c\textbf{u} + c\textbf{v}$ (Left-Distributive Property of Scalar Multiplication over Vector Addition)
        \item $(c+d)\textbf{u} = c\textbf{u} + d\textbf{u}$ (Right-Distributive Property of Scalar Multiplication over Vector Addition)
        \item $c(d\textbf{u}) = (cd)\textbf{u}$
        \item $1\textbf{u} = \textbf{u}$
    \end{enumerate}
\end{theorem}
\begin{theorem} % Unnamed Theorem
	This theorem has no name.
\end{theorem}
\begin{plaintheorem}[ASDF]
	This is extra theorem.
\end{plaintheorem}

\lipsum[1-2]