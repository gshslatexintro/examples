\documentclass[11pt]{report}
\usepackage[left=25mm,right=25mm,top=30mm,bottom=30mm]{geometry}
\usepackage{amsmath} % math
\usepackage{amssymb} % math
\usepackage{graphicx} % to use \includegraphics{}
\usepackage{diagbox} % to make tables
\usepackage{multirow}
\usepackage{caption}
\usepackage{subcaption}
\usepackage{kotex}
\usepackage{color}
\usepackage[hidelinks]{hyperref}
\usepackage[per-mode=symbol]{siunitx}
\sisetup{inter-unit-product =$\cdot$}
\usepackage{titlesec}
\usepackage{amsthm}
\usepackage{datetime}
\usepackage{lipsum}
\usepackage{thmtools}
\title{Linear Algebra Thm Archive - for Mid-Term Exam}
\author{by Gyeonggi Science High School for the Gifted `Linear Algebra' Participants \\
    Main Author : 14129 황동욱 \\
    Revised by : 14121 하석민 \\
    \LaTeX ~ Technician : 14041 박승원
} % 저자 추가하세요.
\date{Last Compilation Time : \today ~ \currenttime}

%%%%%%%%%%%%%%%%%%%%%%%%%%%%
%%%%% 수정하지 마시오. %%%%%
% Theorem formatting
\makeatletter
\newtheoremstyle{GSHScustom} % name
{\topsep}% Space above
{\topsep}% Space below
{}% Body font
{}% Indent amount
{\bfseries}% Theorem head font
{}% Punctuation after theorem head
{\newline} % Space after theorem head
{\thmname{#1}\thmnumber{\@ifnotempty{#1}{ }\@upn{#2}}%
	\thmnote{ {\bfseries {: #3}}}}% Theorem head spec
\makeatother

\makeatletter
\newtheoremstyle{GSHSnonumber} % name
{\topsep}% Space above
{\topsep}% Space below
{}% Body font
{}% Indent amount
{\bfseries}% Theorem head font
{}% Punctuation after theorem head
{\newline} % Space after theorem head
{\thmname{#1}\thmnote{ {\bfseries {: #3}}}}% Theorem head spec
\makeatother


\theoremstyle{GSHScustom}
\newtheorem{theorem}{Theorem}[chapter]

\theoremstyle{GSHSnonumber}
\newtheorem{plaintheorem}{Extra Theorem}[chapter]
\newtheorem{lemma}{Lemma}[chapter]

%% for reducing space above title of chapter
\titleformat{\chapter}[display]
{\normalfont\huge\bfseries}{\chaptertitlename\ \thechapter}{20pt}{\Huge}
\titlespacing*{\chapter}
{0pt}{-20pt}{40pt}
%%%%%%%%% 여기까지 %%%%%%%%


\renewcommand{\labelenumi}{\alph{enumi}.}

\newcommand{\Rn}{$\mathbb{R}^{n}$} % 매번 치기 귀찮아서 아예 명령어를 만들어버림
\newcommand{\Rm}{$\mathbb{R}^{m}$}
\newcommand{\vn}{$ \textbf{v}_1, \textbf{v}_2, \cdots, \textbf{v}_n  $}
\newcommand{\vm}{$ \textbf{v}_1, \textbf{v}_2, \cdots, \textbf{v}_m  $}
\newcommand{\vk}{$ \textbf{v}_1, \textbf{v}_2, \cdots, \textbf{v}_k  $}
\newcommand\inv[1]{#1\raisebox{1.15ex}{$\scriptscriptstyle-\!1$}}

% augmented matrix 만드는 명령어 %
\makeatletter
\renewcommand*\env@matrix[1][*\c@MaxMatrixCols c]{%
	\hskip -\arraycolsep
	\let\@ifnextchar\new@ifnextchar
	\array{#1}}
\makeatother

%\renewcommand{\listtheoremname}{List of Theorems}

\begin{document}
\maketitle
%\tableofcontents
%목차는 딱히 필요없어 보임

% 해당 단원의 tex 파일로 이동하여 편집하세요.
% 참고 : chapter 뿐 아니라 section 마다 이렇게 문서를 나누어 편집하고 싶다는 욕심이 들지도 모르겠지만, 문서가 나뉠 때마다 clearpage 가 되기 때문에 비추천...

\chapter*{Preface and License}
\addcontentsline{toc}{chapter}{Preface and License}
\lipsum[1-2]

Each `Theorem' is identically numbered as textbook. (Except Chapter 3.5) On the other hand, `Extra Theorem' is things that aren't discussed or proved in textbook.

Theorems in this archive can have some errors. Please come to us if you find some of them, then we will revise them.

Anyway, good luck on your mid-term exam on Friday!

This work is licensed under a Creative Commons Attribution-NonCommercial-ShareAlike 4.0 International License.
\begin{figure}[h]
	\centering
	\includegraphics[width=0.4\textwidth]{by-nc-sa.pdf}
\end{figure} \footnote{\url{http://creativecommons.org/licenses/by-nc-sa/4.0/}}
\addcontentsline{toc}{chapter}{Table of Contents}
\tableofcontents
\addcontentsline{toc}{chapter}{List of Theorems}

%\begingroup
%\let\clearpage\relax
\listoftheorems[ignoreall,show={theorem}]
\vspace{2cm}
%\endgroup

\addcontentsline{toc}{chapter}{List of Extra Theorems}
\renewcommand{\listtheoremname}{List of Extra Theorems}

\begingroup
\let\clearpage\relax
\listoftheorems[ignoreall,show={plaintheorem,lemma}]
\endgroup


\chapter{Vectors}

% \iffalse 부터 \fi 까지는 주석처리됨
\iffalse
\begin{theorem}[Algebraic Properties of Vectors in \Rn] \label{properties_vectors}
	Let $\vec{u}$, $\vec{v}$, $\vec{w}$ be vectors in \Rn and let $c$ and $d$ be scalars. Then
	\begin{enumerate}
		\item $\vec{u}+\vec{v}=\vec{v}+\vec{u}$ : Commutativity
		\item $(\vec{u}+\vec{v})+\vec{w}=\vec{u}+(\vec{v}+\vec{w})$ : Associativity
		\item $\vec{u}+\vec{0}=\vec{u}$
		\item $\vec{u}+(-\vec{u})=\vec{0}$
		\item $c(\vec{u}+\vec{v})=c\vec{u}+c\vec{v}$ : Distributivity
		\item $(c+d)\vec{u}=c\vec{u}+d\vec{u}$ : Distributivity
		\item $c(d\vec{u})=(cd)\vec{u}$
		\item $1\vec{u}=\vec{u}$
	\end{enumerate}
\end{theorem}

A theorem shown above is thm.\ref{properties_vectors}.

\begin{theorem}
	Let $\vec{u}$, $\vec{v}$, and $\vec{w}$ be vectors in $\mathbb{R}^{n}$ and let $c$ be a scalar. Then
	\begin{enumerate}
		\item $\vec{u}\cdot\vec{v}=\vec{v}\cdot\vec{u}$
		\item $\vec{u}\cdot(\vec{v}+\vec{w}) = \vec{u}\cdot\vec{v}+\vec{u}\cdot\vec{w}$
		\item $(c\vec{u})\cdot\vec{v}=c(\vec{u}\cdot\vec{v})$
		\item $\vec{u}\cdot\vec{u}\geq 0$ and $\vec{u}\cdot\vec{u}=0$ iff $\vec{u}=\vec{0}$
	\end{enumerate}
\end{theorem}
\begin{proof}
	\begin{equation} \label{pi}
	\pi = 3.141592...
	\end{equation}
	By \eqref{pi}, proved! \qedhere
\end{proof}
\fi

\section{Terminology relating to vectors}

A vector can be represented either in geometric way or algebraic way. In geometric definition of vectors, a vector is a \textbf{directed line segment}.
A vector from point $A$ (\textbf{initial point}, or \textbf{tail}) to point $B$ (\textbf{terminal point}, or \textbf{head}) is denoted as $\overrightarrow{AB}$.
Vectors with their tails in the origin is called \textbf{position vectors}, and they are at \textbf{standard position}.
\\
In algebraic view of vectors, a vector is an \textbf{ordered pair} of \textbf{components}. We denote the set of all vectors containing $n$ components in $\mathbb{R}$ as \Rn. Similarly, set of all vectors containing $n$ integer components is $\mathbb{Z}^{n}$.
\\
A vector is written in form of \textbf{column vectors} and \textbf{row vectors}. We use square brackets for denoting vectors, such as 
\begin{displaymath}
\textbf{v} = \begin{bmatrix}4 \\ 1 \\ 6\end{bmatrix},
\begin{bmatrix} 4, & 1, & 6 \end{bmatrix}
\end{displaymath}
\\
A \textbf{zero vector} is a vector which components are all zero. A zero vector is denoted as $\textbf{0}$.
\\
Two vectors are equal if and only if all the components of two vectors are equal. (Of course, the number of components should be same.)
\\
\textbf{Standard unit Vectors} have components which one of them is 1 and rest of them are all 0. Unit vector which has 1 in $i$th component is denoted as $\textbf{e}_i$, and
\begin{displaymath}
\textbf{e}_i = \begin{bmatrix} 0, & 0, & \cdots & 1, & \cdots & 0 \end{bmatrix}
\end{displaymath}
* \textit{Note} We should always denote vector either with arrows ($\vec{v}$), or with boldface letters ($\textbf{v}$). Scalar denotations ($v$) are not allowed.
\\
A set of vectors with $n$ components taken from finite set of integers $\mathbb{Z}_m = \{0, 1, 2,\cdots,m-1\}$ is denoted as $\mathbb{Z}^{n}_{m}$.
$\mathbb{Z}^{n}_{m}$ is closed with respect to operations of vector addition and scalar multiplication (which is defined later). We can perform those operations in same way, but with modulo operations.
Vectors in $\mathbb{Z}^{n}_2$ (all components are 0 or 1) are called \textbf{binary vectors}.

\section{Basic Operations of Vectors}

\begin{theorem}[Algebraic Properties of Vectors : Basic Vector Operations]
Let \textbf{u},\textbf{v}, and \textbf{w} be vectors in \Rn and let $c$ and $d$ be scalars.
    \begin{enumerate}
        \item $\textbf{u} + \textbf{v} = \textbf{v} + \textbf{u}$ (Commutative Property of Vector Addition)
        \item $(\textbf{u} + \textbf{v}) + \textbf{u} = \textbf{u} + (\textbf{v} + \textbf{w})$ (Associative Property of Vector Addition)
        \item $\textbf{u} + \textbf{0} = \textbf{u}$
        \item $\textbf{u} + (-\textbf{u}) = \textbf{0}$
        \item $c(\textbf{u} + \textbf{v}) = c\textbf{u} + c\textbf{v}$ (Left-Distributive Property of Scalar Multiplication over Vector Addition)
        \item $(c+d)\textbf{u} = c\textbf{u} + d\textbf{u}$ (Right-Distributive Property of Scalar Multiplication over Vector Addition)
        \item $c(d\textbf{u}) = (cd)\textbf{u}$
        \item $1\textbf{u} = \textbf{u}$
    \end{enumerate}
\end{theorem}
\begin{theorem} % Unnamed Theorem
	This theorem has no name.
\end{theorem}
\begin{plaintheorem}[ASDF]
	This is extra theorem.
\end{plaintheorem}

\lipsum[1-2]
\chapter{Systems of Linear Equations}

\section{Terminology}
\textit{Definition.} A vector $\textbf{v}$ is a \textbf{linear combination} of vectors $\textbf{v}_1, \textbf{v}_2, \cdots, \textbf{v}_n$ if there exist scalars $c_1, c_2, \cdots, c_n$ such that
\begin{displaymath}
	c_1\textbf{v}_1 + c_2\textbf{v}_2 + \cdots + c_n\textbf{v}_n = \textbf{0}
\end{displaymath}
The scalars $c_1, c_2, \cdots, c_n$ are called \textbf{coefficients} of linear combination.
\\\\
\textit{Definition.} A \textbf{linear equation} in the $n$ variables $x_1, x_2, \cdots, x_n$ is an equation that can be written in the form of
\begin{displaymath}
	a_1x_1 + a_2x_2 + \cdots + a_nx_n = b
\end{displaymath}
where the \textbf{coefficients} $a_1, a_2, \cdots, a_n$ and the \textbf{constant term} $b$ are constants.
\\\\
A \textbf{set of linear equations} is a finite set of linear equations with same variables. A \textbf{solution} of a linear equation $a_1x_1 + a_2x_2 + \cdots + a_nx_n = b$ is a vector $\begin{bmatrix}
	x_1 & x_2 & \cdots & x_n
\end{bmatrix}$ which satisfies the equation. A solution of a set of linear equations is a vector which is simultaneously a solution of all linear equations in the system. A \textbf{solution set} of a system of linear equations is the set of all solutions of the system.
\\\\
A system of linear equations is \textbf{consistent} if there exists a solution. \textbf{Inconsistent} set of linear equations has an empty solution set.
\noindent Two linear systems are \textbf{equivalent} if they have same solution set.
\\\\
\noindent The \textbf{coefficient matrix} contains the coefficients of variables in the set of linear equations. The \textbf{augmented matrix} is the coefficient matrix augmented by a vector containing constant terms. For the system
\begin{align*}
	\begin{cases}
	a_{11}x_1 + a_{12}x_2 + \cdots + a_{1n}x_n = b_1 \\
	a_{21}x_1 + a_{22}x_2 + \cdots + a_{2n}x_n = b_2 \\
	\cdots \\
	a_{m1}x_1 + a_{m2}x_2 + \cdots + a_{mn}x_n = b_m
	\end{cases}
\end{align*} the coefficient matrix is
\begin{align*}
A = \begin{bmatrix}
	a_{11} & a_{12} & \cdots & a_{1n} \\
	a_{21} & a_{22} & \cdots & a_{2n} \\
	\vdots & \vdots &        & \vdots \\
	a_{m1} & a_{m2} & \cdots & a_{mn}
\end{bmatrix}
\end{align*} and the augmented matrix is
\begin{align*}
	\begin{bmatrix}
		A | \textbf{b}
	\end{bmatrix} = \begin{bmatrix}[cccc|c]
		a_{11} & a_{12} & \cdots & a_{1n} & b_1 \\
		a_{21} & a_{22} & \cdots & a_{2n} & b_2 \\
		\vdots & \vdots &        & \vdots & \vdots \\
		a_{m1} & a_{m2} & \cdots & a_{mn} & b_m \\
	\end{bmatrix}
\end{align*}
\textit{Definition.} A matrix is in \textbf{row echelon form} (REF) if:
\begin{enumerate}
	\item All rows consisting entirely of zeros are at the bottom.
	\item The \textbf{leading entry} (the first nonzero entry) of each rows is located to the left of any leading entries below it.
\end{enumerate}
\textit{Definition.} A matrix is in \textbf{reduced row echelon form} (RREF) if:
\begin{enumerate}
	\item It is in REF.
	\item All leading entries are 1. (\textbf{leading 1})
	\item Each columns containing a leading 1 has 0 everywhere else.
\end{enumerate}

\noindent REF of a matrix is not unique, but all matrices have unique RREF.

\noindent \\ \textit{Definition.} A system of linear equations is \textbf{homogeneous} if the constant term in each equation is zero.

\section{Solving Linear Systems}

\lipsum[3-4]
\chapter{Matrices}
\section{Terminology}

\textit{Definition.} A \textbf{matrix} is a rectangular array of numbers, which are called as \textbf{entries} or \textbf{elements}. If the matrix has $n$ rows and $m$ columns, the \textbf{size} of the matrix is $n \times m$.

\noindent \\ A $1 \times n$ matrix is called a \textbf{row matrix}, or \textbf{row vector}. A $n \times 1$ matrix is called a \textbf{column matrix}, or \textbf{column vector}. (\textit{A vector is considered as a matrix}) We can denote matrices using row vectors or column vectors, such as 
\begin{align*}
A = \begin{bmatrix}
\textbf{A}^{C}_1 & \textbf{A}^{C}_2 & \cdots & \textbf{A}^{C}_m
\end{bmatrix} = \begin{bmatrix}
\textbf{A}^{R}_1 \\ \textbf{A}^{R}_2 \\ \vdots \\ \textbf{A}^{R}_n
\end{bmatrix}
\end{align*} where $\textbf{A}^{C}_i$ is the $i$th column of $A$ and $\textbf{A}^{R}_i$ is the $i$th row of $A$.

\noindent \\ The element at $i$th row and $j$th column is denoted by $A_{ij}$. We can also denote matrices using elements, such as $A = [A_{ij}]$.

\noindent \\ \textit{Definition.} The \textbf{diagonal entries}of $A$ are $A_{ii}$.

\noindent \\ \textit{Definition.} The \textbf{square matrix} is a matrix which has same number of rows and columns (so the size is $n \times n$). \textbf{Diagonal matrix} is a square matrix which has its nondiagonal entries as 0. A diagonal matrix with all of its diagonal entries are the same are \textbf{scalar matrix}. If the value of diagonal entries are all 1, it is \textbf{identity matrix}.

\noindent A $n \times n$ identity matrix is denoted as $I_n$, and
\begin{align*}
I_n = \begin{bmatrix}
1 & 0 & \cdots & 0 \\
0 & 1 & \cdots & 0 \\
\vdots & \vdots & & \vdots \\
0 & 0 & \cdots & 1
\end{bmatrix}
\end{align*}

\noindent A \textbf{zero matrix} $O$ is a matrix which all of its entires are zero.

\noindent \\ \textit{Definition.} Two matrices are \textbf{equal} if and only if

(i) The size of two matrices are the same.

(ii) The corresponding entries of the matrices are the same.

\section{Matrix Operations}

\lipsum[5-6]
%\include{Ch4_eigenvalues_and_eigenvectors}
\appendix
\chapter{Cautions on Exam}
\section{Notations}
\begin{table}[h]
	\centering
	\begin{tabular}{|c|c|c|}
		\hline
		Wrong & Right & Explanation \\
		\hline
		$ 2 \cdot 3 = 6 $ & $ 2 \times 3 = 6 $ & \multirow{2}{*}{$ \cdot $ for dot product is only valid for vectors.} \\
		\cline{1-2}
		$ 2 \cdot \textbf{v}$ & $2\textbf{v} $ & \\
		\hline
		$\begin{bmatrix} 1 & 3 & 4 \end{bmatrix}$ & $\begin{bmatrix} 1, & 3, & 4 \end{bmatrix}$ & When writing row vectors, commas are necessary. \\
		\hline
		
	\end{tabular}
\end{table}
\section{Description}
\begin{itemize}
\item Free variables like $s$, $t$ and $u$ must be indicated that they are arbitrary real number. \textit{e.g.} $ s,t \in \mathbb{R} $
%\item Whenever you use FTIM, always start by writing `according to fundamental theorem of invertible matrices' or `가역행렬의 기본정리에 의해'.
\item When proving $ \text{span}\left(\textbf{v}_1,\textbf{v}_2,\textbf{v}_3\right) = \mathbb{R}^{3} $, you must prove each side, not only one.
\item $ A=B $ means not only same entries, but also same size.
%\item Do not write reduced row echelon form as `ref'. This abbreviation is not defined in our book.
\item Write as `적어도 하나는 0이 아닌' instead of `모두 0은 아닌' or else.
\item Abbreviation : only 4 things are allowed : `REF', `RREF', 'EMO', 'F.T.I.M.'. Each of them stands for (Reduced) Row Echelon Form, Elementary Matrix Operation, Fundamental Theorems of Invertible Matrices.
\end{itemize}

\end{document}
